\documentclass{capstone}
\usepackage[T1]{fontenc}
\usepackage{parskip}
\usepackage{geometry}
\geometry{margin=2.5cm}
\usepackage{bookman}
\usepackage{unnumberedtotoc}
\usepackage{graphicx}
\usepackage[font={small,it}]{caption}
\usepackage{array}
\usepackage{subcaption}

\begin{document}

\titlepage
{\includegraphics[width=\paperwidth]{assets/ecse-decal-title}}
{
	\centering
	{\Large ECSE Capstone Project Proposal\par}
	\vspace{16pt}
	{Anonymous\par}
	\vspace{16pt}
	{Semester 1, 2025\par} 	% Replace YEAR with the current year
}

\tableofcontents
\newpage

\addsec{Glossary of Terms}

\begin{itemize}
	\item \textbf{ADC}: Analog-to-Digital Converter
	\item \textbf{CRUD}: Create, Read, Update, Delete
	\item \textbf{ECU}: EVolocity Control Unit
	\item \textbf{LED}: Light Emitting Diode
	\item \textbf{MCU}: Micro Controller Unit
	\item \textbf{MVP}: Minimum Viable Product
	\item \textbf{PCB}: Printed Circuit Board
	\item \textbf{PoC}: Proof of Concept
	\item \textbf{RPi}: Raspberry Pi
	\item \textbf{SMD}: Surface Mount Device
	\item \textbf{SMT}: Surface Mount Technology
	\item \textbf{SMPS}: Switch Mode Power Supply
\end{itemize}

\newpage

\addsec{Introduction}

This project addresses an operational challenge faced by EVolocity, an organization that hosts competitions for primary, intermediate, and high school students to design and race electric vehicles.
Currently, EVolocity officials spend a lot of time manually collecting performance data from each vehicle's control unit, creating delays in race-day operations.

Our solution implements a comprehensive wireless data collection system that integrates hardware, firmware, and software components.
The hardware team is developing a custom PCB with voltage and current sensing capabilities along with wireless transmission functionality.
The firmware team is creating resilient data acquisition, storage, and transmission protocols to ensure reliable measurements even in the challenging race environment.
Lastly, the software team is building a relational database with an intuitive web interface that allows officials to efficiently manage competitions, teams, and real-time performance metrics.

By eliminating manual data collection, this system will reduce operational overhead, improve data accuracy, and enhance the overall competition experience.

\addsec{Problem Statement}

At EVolocity competitions, race officials retrieve data from the ECU of each vehicle manually by connecting to each one via USB-C.
The current implementation is time consuming and inefficient, particularly in fast-paced environments where quick data access is critical.
End users, including race officials and event organisers, rely on this data to validate results and ensure fair competition.
A wireless system to collect ECU data is needed to streamline this process and improve overall event efficiency.

\addsec{Proposed Solution}

Our solution will provide a wireless data collection system to retrieve, store, and display ECU data. The overall system architecture is shown in Figure~\ref{fig:basic_system_diagram}.

The hardware consists of a custom-designed PCB that works together with an embedded RPi Pico W microcontroller to transmit and validate data to enable scoring, event tracking, and race management.
The firmware, running on the RPi Pico W, handles wireless transmission of data between the ECU and the backend server ensuring the data is reliably validated and timestamped.
The software includes a backend server and a web-based interface that enables race officials to view scoring, monitor energy efficiency and manage competition data.

Compared to the current manual method, our system reduces data retrieval time and improves operational efficiency. Key advantages include access to race metrics, and a streamlined user-friendly dashboard tailored to race-day workflows.

\begin{figure}[h!]
	\centering
	\includegraphics[width=0.6\textwidth]{assets/basic_sys_diagram}
	\caption{Basic high-level system architecture diagram.}
	\label{fig:basic_system_diagram}
\end{figure}

\pagebreak

\addsec{MVP Scope \& Features}

% This section outlines core functionality and essential features required for a viable product.

\subsection*{Hardware}

\subsubsection*{Objectives}

\begin{itemize}
	\item Develop an energy monitor with voltage and current sensing capabilities for the race vehicle.
	\item Demonstrate the core functionality of the system through a simplified PoC.
\end{itemize}

\subsubsection*{Must Have}

\begin{itemize}
	\item Voltage \& current sensors.
	      \begin{itemize}
		      \item Accuracy: ±1\% without calibration per unit.
		      \item Precision: ±5\% when taken as an average per batch.
	      \end{itemize}
	\item Power regulation (12V to 5V and 3V conversion).
	\item Wireless transceiver (for ECU data transmission).
	\item Basic power indicator LEDs.
\end{itemize}

% \subsubsection*{Should Have}

\subsubsection*{Could Have}

\begin{itemize}
	\item Debug/Status LEDs to show system status.
	\item Custom buck converter instead of using linear regulator to improve power efficiency.
	\item Buffer (voltage follower) for the voltage sensing circuit.
	\item Reverse polarity protection and TVS diodes or zener diodes on input and outputs for ESD and over-voltage protection.
	\item Better schematic layout to:
	      \begin{itemize}
		      \item Avoid current loops (causes magnetic fields to be induced, resulting in more noise).
		      \item Avoid long ground traces and make 5V regulator to sink paths more straight/direct.
	      \end{itemize}
\end{itemize}

\subsubsection*{Would Have}

\begin{itemize}
	\item Higher quality operational amplifier (e.g. LT6221 -- \$14.70/unit) - better signal conditioning.
\end{itemize}

\subsection*{Firmware}

\subsubsection*{Objectives}

\begin{itemize}
	\item Collect ECU data from sensors and transmit it wirelessly using the RPi Pico W.
	\item Ensure accurate data collection, robust transmission, and secure temporary storage.
\end{itemize}

\subsubsection*{Must Have}

\begin{itemize}
	\item State machine, all tasks done sequentially with no concurrency.
	      \begin{itemize}
		      \item Avoid Wi-Fi uploading whilst sampling from ADC to improve accuracy.
	      \end{itemize}
	\item ADC sampling for voltage \& current readings.
	      \begin{itemize}
		      \item ±1\% system accuracy (from the hardware to the GUI).
	      \end{itemize}
	\item Wireless data transmission.
	      \begin{itemize}
		      \item Minimum range: 10 meters. Automatic retries until the connection succeeds.
		      \item Less than 5 minutes to transfer 1 hour of data
	      \end{itemize}
	\item Sampling interval: minimum of once every 5 seconds.
	\item Ensure data is properly received and transmitted.

\end{itemize}

\subsubsection*{Should Have}

\begin{itemize}
	\item Wear-leveling for flash memory, or a better alternative would be support for SD card storage to extend logging capacity.
	      \begin{itemize}
		      \item Spread out writes across all parts of the NAND chip to prolong overall lifespan of the ROM.
		      \item Better data handling in the event of power loss (e.g. better resilience against corruption compared to onboard flash).
	      \end{itemize}
	\item Use efficient formats like Protocol Buffers instead of JSON.
\end{itemize}

\subsubsection*{Could Have}

\begin{itemize}
	\item Integrate alternative modules to account for the shortcomings of RPi Pico W.
	      \begin{itemize}
		      \item e.g. ESP32 to improve wireless range and speed.
		      \item Dedicated ADC peripheral or another microcontroller with a more accurate ADC, e.g. ATmega328P/ATmega328PB.
	      \end{itemize}
\end{itemize}

% \subsubsection*{Would Have}

\subsection*{Software}

\subsubsection*{Objectives}

\begin{itemize}
	\item Provide a user-friendly interface for managing and visualising ECU data, teams, events, and race results.
	\item Enable wireless data retrieval from ECUs.
	\item Allow offline data storage on competition day, with later syncing to an online database when the user has internet access.
\end{itemize}

\subsubsection*{Must Have}

\begin{itemize}
	\item \textbf{User Interface}
	      \begin{itemize}
		      \item Simple and functional design (internal tooling, not a public-facing site).
		      \item Mobile-friendly views for all pages.
		      \item Competitions Page. (CRUD operations for competitions).
		            \begin{itemize}
			            \item Each competition has three events by default.
			            \item Teams can be assigned to any number of events.
		            \end{itemize}
		      \item Events and Event Types Page. (CRUD operations for events \& event types).
		      \item Scoreboard.
		      \item Teams Page. (CRUD operationsfor teams).
		      \item Energy Monitors Page.
		            \begin{itemize}
			            \item Display ECU properties (timestamp, temperature, average voltage, average current, energy).
			            \item Energy graphs.
		            \end{itemize}
		      \item ECU configuration \& multi-race support.
	      \end{itemize}
	\item \textbf{Backend}
	      \begin{itemize}
		      \item Local area network (LAN) connection to ECUs.
		      \item API endpoints for ECU data uploads.
		      \item Automatic team scoring.
	      \end{itemize}
	\item \textbf{Integration}
	      \begin{itemize}
		      \item Frontend-backend integration using Next.js (Server Actions as middleware).
		      \item Warnings for corrupt or conflicting data.
		      \item Time and data synchronisation with ECUs.
	      \end{itemize}
\end{itemize}

\subsubsection*{Should Have}

\begin{itemize}
	\item CRUD operations for wheel configurations.
	\item TLS encryption for API endpoints.
\end{itemize}

\subsubsection*{Could Have}

\begin{itemize}
	\item Real-time viewing using a real-time database (e.g. Supabase).
	\item Battery status monitoring from the interface.
	\item User authorisation for using the tool (auth system).
\end{itemize}

% \pagebreak

\addsec{Project Plan}

This section outlines our project strategy, including deliverables, milestones indicated by Figure~\ref{fig:timeline}. Our approach ensures efficient resource allocation and clear responsibility assignments to meet the project requirements within the given timeframe.

\subsection*{Implementation Timeline}

\begin{figure}[h!]
	\centering
	\includegraphics[width=0.75\textwidth]{assets/simplified_gantt}
	\caption{Project timeline Gantt chart showing key milestones, task dependencies, and estimated completion dates. For a more detailed breakdown of the tasks and deliverables, please refer to Appendix A and Appendix B.}
	\label{fig:timeline}
\end{figure}

\addsec{Technical Design}

\subsection*{System Overview}

The system integrates custom hardware (PCB with RPi Pico W), firmware for data handling, and web-based software to enable wireless collection and display of ECU data. The components work together to provide reliable data transmission, storage, and visualization for race officials as indicated in Figure~\ref{fig:full_system_diagram}.

\begin{figure}[h!]
	\centering
	\includegraphics[width=0.75\textwidth]{assets/full_sys_diagram}
	\caption{Detailed system architecture diagram showing the interaction between hardware, firmware, and software components.}
	\label{fig:full_system_diagram}
\end{figure}

\subsection*{Hardware}

\subsubsection*{Tools \& Technologies}

\begin{itemize}
	\item Simulation: LTspice
	\item PCB Design: Altium Designer
\end{itemize}

\subsubsection*{Component Selection}

Linear Regulator (L78L05): Converts 12V to 5V for the RPi Pico W, which requires 1.8-5.5V input. The 5V feeds into the Pico's built-in SMPS, which generates 3.3V for the RP2040 microcontroller. Originally planned to use LD1117 due to local availability, but switched to L78L05 due to Altium library constraints.

ADC Reference: Implemented with a voltage divider by using resistors in parallel providing a stable 3V reference. This reduces electrical noise from the RPi's built-in SMPS, which can generate interference due to magnetic and switching characteristics. This approach was more cost-effective than using high-capacitance electrolytic capacitors.

Power LED: Provides essential visual feedback that confirms the PCB is receiving power and operating correctly. This visual feedback is particularly important during races when no testing equipment is available on the track.

Microcontroller: RPi Pico W selected for its wireless communication capabilities. Power is supplied through either $V_{bus}$ (USB port) or $V_{sys}$ (external power) pins. When using both power sources simultaneously, Schottky diodes are arranged in an 'OR'ing configuration to allow seamless power switching when programming.

\subsubsection*{Bill of Materials}
Table~\ref{tab:BOM} below shows the prices from PCBWay \cite{pcbway}, DigiKey \cite{cap_electrolytic} \cite{shunt_resistor} \cite{led} \cite{opamp} \cite{linear_regulator} \cite{pico_w} \cite{resistors} \cite{ceramic_caps} \cite{test_points} and Element14 \cite{schottky_diode} for the components used when ordered in bulk to manufacture 50 ECUs.

\begin{table}[h!]
	\centering
	\renewcommand{\arraystretch}{1.5}
	\caption{Bill of Materials (BOM) for the ECU hardware components.}
	\begin{tabular}{|p{0.1\textwidth}|p{0.25\textwidth}|p{0.2\textwidth}|p{0.05\textwidth}|p{0.3\textwidth}|}
		\hline
		\textbf{Item No.} & \textbf{Component}      & \textbf{Part Number} & \textbf{Qty} & \textbf{Unit Cost} \\
		\hline
		1                 & Op-Amp                  & LM324DR2G            & 1            & \$0.64             \\
		\hline
		2                 & Shunt Resistor          & ER74R10KT            & 2            & \$1.80             \\
		\hline
		3                 & Linear Regulator        & L78L05               & 1            & \$1.37             \\
		\hline
		4                 & Schottky Diode          & SP304                & 1            & \$0.95             \\
		\hline
		5                 & LED                     & LTST-C191KRKT        & 1            & \$0.20             \\
		\hline
		6                 & RPi Pico W              & SC0918               & 1            & \$11.59            \\
		\hline
		7                 & Test points             & 5003                 & 5            & \$0.26             \\
		\hline
		8                 & Electrolytic Capacitors & TH                   & 1            & \$0.26             \\
		\hline
		9                 & Ceramic Capacitors      & 0805 SMT             & 5            & \$0.02             \\
		\hline
		10                & Resistors               & 0805 SMT             & 11           & \$0.06             \\
		\hline
		11                & PCB                     & -                    & 1            & \$5.00             \\
		\hline
		                  &                         &                      & Total        & \$25.65            \\
		\hline
	\end{tabular}
	\label{tab:BOM}
\end{table}

\subsubsection*{Circuit Schematic}

Current sensing uses a 3W-rated 0.1$\Omega$ shunt resistor to handle the peak motor current of 3A, as standard 100mW SMT resistors would be insufficient.
Voltage sensing uses two resistors (10k$\Omega$ and 1.5k$\Omega$) in series, with values chosen in the k$\Omega$ range to minimize power dissipation while staying below 100k$\Omega$ to reduce thermal noise and environmental sensitivity.
Figure~\ref{fig:sensing_circuit} displays both the current and voltage sensor schematic. An overall schematic of the whole monitor can also be viewed in Appendix C and D.

\begin{figure}[h!]
	\centering
	\includegraphics[width=0.37\textwidth]{assets/hardware/sensing_circuit}
	\caption{LTSpice circuit schematic showing the voltage and current sensing circuitry.}
	\label{fig:sensing_circuit}
\end{figure}

Decoupling capacitors are placed near to power sinks to resist against poor wire connections and extend the duration before brownout occurs, allowing more time for the RPi to react.

\subsubsection*{PCB Layout}

Our initial PCB layout shown in Figure~\ref{fig:pcb_top_layer} was designed to seperate high-power, high-frequency (right side), and low-power signal processing components (left side) to reduce noise and interference. As the first prototype, it was also designed to be easy to solder and have good heat disappation characteristics. Optimization of the PCB for cost in a future revision, as the current design is not cost-effective for mass production. The high power connections, such as the input, output, shunt resistors and regulator are placed close to each other and connected via power planes to reduce resistance and noise generation. By using a 5V and Ground plane, we could take advantage of a little inter-plane capacitance.

\begin{figure}[h!]
	\centering
	\includegraphics[width=0.55\textwidth]{assets/hardware/pcb_top_layer.png}
	\caption{Altium Designer screenshot shoting Top Layer of PCB layout.}
	\label{fig:pcb_top_layer}
\end{figure}

\subsection*{Firmware}

\subsubsection*{Tools \& Technologies}

\begin{itemize}
	\item Programming Language: C/C++
	\item Wi-Fi for connecting to the database since long uninterrupted range.
	\item Hardware dependencies: RPi Pico Integration with PCB and electrical components for sampling.
\end{itemize}

We selected C/C++ for optimized performance and direct hardware control, enabling faster server discovery and maximizing flash storage for offline data collection. This allows complete event operation even without main server connectivity.

Our firmware architecture (Figure~\ref{fig:firmware_system_diagram}) separates ADC sampling from wireless transmission to prevent power fluctuations that would affect measurement accuracy, especially important due to the RP2040's documented non-linear ADC behavior (errata RP2040-E11 \cite{rp2040_datasheet}). We implement a microcontroller-specific calibration curve to compensate for this non-linearity.

Data management uses a chunked storage approach that optimizes both memory usage and network efficiency. Key features include:
\begin{itemize}
	\item Round-robin transmission scheduling to prevent server overload
	\item JSON for ease of implementation with CRC32 checksums for data integrity verification
	\item Brownout detection to prevent data corruption during power failure
\end{itemize}

\subsubsection*{Data Storage Plan}

Our firmware implements an efficient, resilient data management strategy:

\begin{itemize}
	\item Configuration Persistence: Stores last connected server IP address for immediate reconnection on startup.
	\item Device Identification: Generates and maintains a unique identifier for each Pico device.
	\item Session Management: Creates a new "run" ID on each power cycle for organized data collection.
	\item Measurement Store: Records key measurements with timestamps in reasonable chunks.
	\item Optimized Transfer: Implements batch uploading to preserve limited RAM while ensuring data integrity.
\end{itemize}

\subsubsection*{Firmware System Diagram}

\begin{figure}[h!]
	\centering
	\includegraphics[width=0.65\textwidth]{assets/firmware_technical}
	\caption{Firmware system architecture showing flow and key components.}
	\label{fig:firmware_system_diagram}
\end{figure}

\subsection*{Software}

\subsubsection*{Tools \& Technologies}

\begin{itemize}
	\item Frontend: Next.js + Tailwind
	\item Component Library: ShadCN
	\item Backend: Next.js + Prisma
	\item Database: PostgreSQL, Prisma builtin adapter available (Figure~\ref{fig:database_schema}).
\end{itemize}

\subsubsection*{Database Schema}

\begin{figure}[h!]
	\centering
	\includegraphics[width=0.75\textwidth]{assets/database_schema}
	\caption{Database schema showing relationships between competitions, teams, events, and ECU data.}
	\label{fig:database_schema}
\end{figure}

\subsubsection*{User Workflow}

The system implements an intuitive, step-by-step workflow that guides race officials through the process of creating competitions, registering teams, and collecting data (Figure~\ref{fig:user_workflow}).

\begin{figure}[h!]
	\centering
	\includegraphics[width=0.75\textwidth]{assets/user_flow}
	\caption{User workflow showing the process of registering a team, creating a competition, and uploading data.}
	\label{fig:user_workflow}
\end{figure}

\subsubsection*{UI/UX Design}

Our user interface is designed for simplicity and efficiency, with intuitive navigation and clear information presentation as seen in Figure~\ref{fig:ui_competitions} and Figure~\ref{fig:ui_teams}.

\begin{figure}[h!]
	\centering
	\begin{subfigure}[b]{0.48\textwidth}
		\includegraphics[width=\textwidth]{assets/ui_prototype/landing_page.png}
		\caption{Landing page dashboard with system overview.}
		\label{fig:landing_page}
	\end{subfigure}
	\hfill
	\begin{subfigure}[b]{0.48\textwidth}
		\includegraphics[width=\textwidth]{assets/ui_prototype/competitions_1.png}
		\caption{Competitions list view showing all events.}
		\label{fig:competitions_1}
	\end{subfigure}

	\vspace{0.5cm}

	\begin{subfigure}[b]{0.48\textwidth}
		\includegraphics[width=\textwidth]{assets/ui_prototype/competitions_2.png}
		\caption{Competition details with team assignments.}
		\label{fig:competitions_2}
	\end{subfigure}
	\hfill
	\begin{subfigure}[b]{0.48\textwidth}
		\includegraphics[width=\textwidth]{assets/ui_prototype/competitions_3.png}
		\caption{Event configuration with scoring parameters.}
		\label{fig:competitions_3}
	\end{subfigure}

	\caption{Web interface showing competition management features with clean, functional design.}
	\label{fig:ui_competitions}
\end{figure}

\begin{figure}[h!]
	\centering
	\begin{subfigure}[b]{0.48\textwidth}
		\includegraphics[width=\textwidth]{assets/ui_prototype/competitions_4.png}
		\caption{Real-time competition scoring view.}
		\label{fig:competitions_4}
	\end{subfigure}
	\hfill
	\begin{subfigure}[b]{0.48\textwidth}
		\includegraphics[width=\textwidth]{assets/ui_prototype/teams_1.png}
		\caption{Teams management interface.}
		\label{fig:teams_1}
	\end{subfigure}

	\vspace{0.5cm}

	\begin{subfigure}[b]{0.48\textwidth}
		\includegraphics[width=\textwidth]{assets/ui_prototype/teams_2.png}
		\caption{Team detail view with member information.}
		\label{fig:teams_2}
	\end{subfigure}
	\hfill
	\begin{subfigure}[b]{0.48\textwidth}
		\includegraphics[width=\textwidth]{assets/ui_prototype/team_3.png}
		\caption{Energy monitoring dashboard for a team.}
		\label{fig:teams_3}
	\end{subfigure}

	\caption{Team management and energy monitoring interfaces.}
	\label{fig:ui_teams}
\end{figure}

\addsec{Initial Testing Plan}

The testing plan will be divided into three main phases: hardware, firmware, and software testing.

For simulation, LTSpice will be used for simulating the expected values and LD1117 regulator was used in testing to provide a stable stepped down 5V DC voltage to the system. From our testing results in Figure~\ref{fig:test_measurements}, we can see that this regulator has some noise, to mitigate this we have used smoothing capacitors to reduce the noise. A small decoupling capacitor in uF range could be used to further reduce the noise.

From testing (Figure~\ref{fig:test_measurements}), we can see that we have successfully implemented a stable 3V ADC reference with minimal noise which may be used as an alternative to the internal reference voltage generated by the SMPS onboard the RPi.

As for our sensing circuitry, the LM324 opamp was used to minimise the costs whilst still providing good functionality. However, this comes at the cost of added noise to the output of the opamp which we can clearly observe in the test results. The characteristics of the opamp such as a relatively high input offset voltage (5 mV), and input bias current (40 uA) all contribute to noise. In addition, Johnson noise is introduced by the resistors in the amplifier. The combined thermal noise from R1 to R4 can add noise to the output. In practical testing, Vvo was higher than the simulated output. This is mainly due to the shunt resistor tolerance differences. Ideally, we want a 0.1 $\Omega$ shunt resistor, but practically this is closer to 0.3 $\Omega$. The increased shunt resistance caused an increase in Vvo.

\begin{figure}[h!]
	\centering
	\begin{subfigure}[b]{0.48\textwidth}
		\centering
		\includegraphics[width=\textwidth]{assets/testing/5V_ref.png}
		\caption{Noise characteristics of the 5V power rail with the linear regulator.}
		\label{fig:5v_ref}
	\end{subfigure}
	\hfill
	\begin{subfigure}[b]{0.48\textwidth}
		\centering
		\includegraphics[width=\textwidth]{assets/testing/3V_ref.png}
		\caption{3V reference voltage for the ADC.}
		\label{fig:3v_ref}
	\end{subfigure}
	\vspace{0.5cm}

	\begin{subfigure}[b]{0.48\textwidth}
		\centering
		\includegraphics[width=\textwidth]{assets/testing/v_v0.png}
		\caption{Voltage as probed at the Vvo pin.}
		\label{fig:v_v0}
	\end{subfigure}
	\hfill
	\begin{subfigure}[b]{0.48\textwidth}
		\centering
		\includegraphics[width=\textwidth]{assets/testing/current_sensor.png}
		\caption{Noise characteristics of the current sensing circuit.}
		\label{fig:current_sensor}
	\end{subfigure}

	\vspace{0.5cm}

	\caption{Oscilloscope measurements of various circuit components.}
	\label{fig:test_measurements}
\end{figure}

\addsec{Conclusion}

Our wireless data collection system directly addresses EVolocity's need for efficient performance data retrieval from competition vehicles. The proposed solution integrates:

\begin{itemize}
	\item An affordable custom PCB with precise voltage and current sensing circuitry, validated through comprehensive testing
	\item Firmware optimized for reliable data collection, storage, and transmission in race conditions
	\item A user-friendly web interface enabling officials to manage competitions and analyze performance metrics
\end{itemize}

Through extensive collaboration between hardware, firmware, and software teams, we've developed an affordable and technically viable solution that can be implemented within the specified timeline. By eliminating manual data collection processes, this system will enhance the competition experience for both organizers and participants by providing real-time access to critical performance data while reducing operational overhead.

\pagebreak

\begin{thebibliography}{9}
	\raggedright
	\bibitem{rp2040_datasheet}
	Raspberry Pi Ltd, ``RP2040 Datasheet,'' Raspberry Pi Ltd, Cambridge, UK, 2021. [Online]. Available: https://datasheets.raspberrypi.com/rp2040/rp2040-datasheet.pdf

	\bibitem{cap_electrolytic}
	Nichicon, ``UVR Series Aluminum Electrolytic Capacitors,'' Nichicon Corporation, Japan, 2023. [Online]. Available: https://www.digikey.co.nz/en/products/detail/nichicon/UVR2C100MPD/588902

	\bibitem{shunt_resistor}
	TE Connectivity, ``ER Series Current Sense Resistors,'' TE Connectivity Ltd., Switzerland, 2022. [Online]. Available: https://www.digikey.co.nz/en/products/detail/te-connectivity-passive-product/ER74R10KT/2365306

	\bibitem{schottky_diode}
	Taiwan Semiconductor, ``SR304 Schottky Barrier Rectifier,'' Taiwan Semiconductor Co., Ltd., Taiwan, 2022. [Online]. Available: https://nz.element14.com/taiwan-semiconductor/sr304-r0/diode-schottky-3a-40v/dp/7278403

	\bibitem{led}
	Lite-On, ``LTST-C191KRKT SMD LED,'' Lite-On Technology Corporation, Taiwan, 2022. [Online]. Available: https://www.digikey.co.nz/en/products/detail/liteon/LTST-C191KRKT/386837

	\bibitem{opamp}
	ON Semiconductor, ``LM324 Low Power Quad Operational Amplifier,'' ON Semiconductor, USA, 2022. [Online]. Available: https://www.digikey.co.nz/en/products/detail/onsemi/LM324DR2G/918511

	\bibitem{linear_regulator}
	STMicroelectronics, ``L78L05ABUTR Voltage Regulator,'' STMicroelectronics, Switzerland, 2022. [Online]. Available: https://www.digikey.co.nz/en/products/detail/stmicroelectronics/L78L05ABUTR/\newline585705

	\bibitem{pcbway}
	PCBWay, ``PCB Assembly Services,'' PCBWay, China, 2023. [Online]. Available: https://www.pcbway.com

	\bibitem{pico_w}
	Raspberry Pi Ltd, ``Raspberry Pi Pico W,'' Raspberry Pi Ltd, UK, 2022. [Online]. Available: https://www.digikey.co.nz/en/products/detail/raspberry-pi/SC0918/16608263

	\bibitem{resistors}
	Yageo, ``RC Series Chip Resistors,'' Yageo Corporation, Taiwan, 2022. [Online]. Available: https://www.digikey.co.nz/en/products/detail/yageo/RC0805FR-0710KL/727535

	\bibitem{ceramic_caps}
	Samsung Electro-Mechanics, ``CL21 Series Ceramic Capacitors,'' Samsung Electro-Mechanics, South Korea, 2022. [Online]. Available: https://www.digikey.co.nz/en/products/detail/samsung-electro-mechanics/CL21B104KBCNNNC/3886661

	\bibitem{test_points}
	Keystone Electronics, ``PCB Test Points,'' Keystone Electronics Corp., USA, 2022. [Online]. Available: https://www.digikey.co.nz/en/products/detail/keystone-electronics/5003/362668

\end{thebibliography}

\pagebreak

\addsec{Appendices}

\subsection*{Appendix A: Deliverables and Milestones}

\begin{table}[h!]
	\centering
	\renewcommand{\arraystretch}{1.5}
	\caption{Project milestones and deliverables with their due dates and current status.}
	\begin{tabular}{|p{0.3\textwidth}|>{\raggedright\arraybackslash}p{0.3\textwidth}|>{\raggedright\arraybackslash}p{0.3\textwidth}|}
		\hline
		\textbf{Item}    & \textbf{Due Date} & \textbf{Status} \\
		\hline
		Risk Analaysis   & 14th March 2025   & Submitted       \\
		\hline
		PCB Submission   & 11th April 2025   & Submitted       \\
		\hline
		Project Proposal & 14th April 2025   & Done            \\
		\hline
		Final Report     & 23rd May 2025     & Upcoming        \\
		\hline
		Demonstration    & 26th May 2025     & Upcoming        \\
		\hline
	\end{tabular}
	\label{tab:deliverables}
\end{table}

\subsection*{Appendix B: Task Breakdown}

\begin{table}[h!]
	\centering
	\renewcommand{\arraystretch}{1.5}
	\caption{Breakdown of project tasks and responsibilities assigned to team members by discipline.}
	\begin{tabular}{|p{0.15\textwidth}|>{\raggedright\arraybackslash}p{0.5\textwidth}|>{\raggedright\arraybackslash}p{0.25\textwidth}|}
		\hline
		\textbf{Discipline} & \textbf{Task}                            & \textbf{Assigned To} \\
		\hline
		Hardware            & Implementation \& verification           & V                    \\
		\hline
		Hardware            & PCB design                               & A \& T               \\
		\hline
		Hardware            & Soldering \& assembly                    & A                    \\
		\hline
		Hardware            & Testing \& prototyping                   & J                    \\
		\hline
		Hardware            & Demo with RPi Pico W                     & A \& J \& V          \\
		\hline
		Firmware            & ADC implementation \& sampling           & S                    \\
		\hline
		Firmware            & Data validation \& error checking        & S \& T               \\
		\hline
		Firmware            & Flash memory storage \& implementation   & T                    \\
		\hline
		Firmware            & Wi-Fi connection                         & P \& T               \\
		\hline
		Firmware            & HTTP(S) data transmission                & P                    \\
		\hline
		Firmware            & Ensure successful data transfer          & P                    \\
		\hline
		Software            & Backend - Database \& Schema             & Z                    \\
		\hline
		Software            & Backend - Server \& API setup            & J                    \\
		\hline
		Software            & Frontend - Teams \& Energy Monitor pages & C                    \\
		\hline
		Software            & Frontend - Competitions \& Events pages  & C                    \\
		\hline
	\end{tabular}
	\label{tab:task_breakdown}
\end{table}

\pagebreak

\subsection*{Appendix C: LTSpice Schematic}

\begin{figure}[h!]
	\centering
	\includegraphics[width=0.75\textwidth]{assets/hardware/energy_monitor.png}
	\caption{LTSpice schematic of the circuit.}
	\label{fig:ltspice_schematic}
\end{figure}

\subsection*{Appendix D: Altium Schematic}

\begin{figure}[h!]
	\centering
	\includegraphics[width=0.75\textwidth]{assets/hardware/altium_schematic.png}
	\caption{Altium schematic of the circuit.}
	\label{fig:altium_schematic}
\end{figure}

\pagebreak

\subsection*{Appendix E: PCB Bottom Layout}

\begin{figure}[h!]
	\centering
	\includegraphics[width=0.75\textwidth]{assets/hardware/altium_bottom_layer.png}
	\caption{Bottom layer of PCB layout of the circuit.}
	\label{fig:altium_bottom_layer}
\end{figure}

\subsection*{Appendix F: PCB Render}

\begin{figure}[h!]
	\centering
	\includegraphics[width=0.75\textwidth]{assets/hardware/altium_render.png}
	\caption{Altium 3D view of PCB layout of the circuit.}
	\label{fig:altium_render}
\end{figure}

\end{document}